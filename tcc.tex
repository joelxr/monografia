\documentclass[12pt,twoside,openright,a4paper,english,brazil,sumario=tradicional]{abntex2}
\usepackage{lmodern}
\usepackage[utf8]{inputenc}
\usepackage[T1]{fontenc}
\usepackage{indentfirst}
\usepackage{color}
\usepackage{graphicx}
\usepackage{microtype}
\usepackage[brazilian,hyperpageref]{backref}
\usepackage[alf]{abntex2cite}
\usepackage{multicol}
\usepackage{multirow}
\usepackage{enumitem}
\usepackage{amsfonts}
\usepackage{amssymb}
\usepackage{listings}
\usepackage{tikz-uml}

\graphicspath{{../images/}}

\renewcommand{\backrefpagesname}{Citado na(s) página(s):~}
\renewcommand{\backref}{}
\renewcommand*{\backrefalt}[4]{
	\ifcase #1 %
		Nenhuma citação no texto.%
	\or
		Citado na página #2.%
	\else
		Citado #1 vezes nas páginas #2.%
	\fi}%

\titulo{Módulo de pré-vizualização para a ferramenta CGT - Ceará Game Tools}
\author{Joel Xavier Rocha\thanks{\texttt{joelxr@gmail.com}}}
\orientador{Prof. Dr. Carlos Hairon Ribeiro Gonçalves}
\local{Fortaleza}
\data{2015}
\instituicao{
  Instituto Federal de Ciência, Educação e Tecnologia do Ceará
  \par
  Engenharia de Computação}
\tipotrabalho{Monografia}
\preambulo{Trabalho de conclusão de curso apresentado à banca examinadora do Instituto Federal de Educação, Ciência e Tecnologia do Ceará para obtenção do grau de bacharel em Engenharia de Computação sob a orientação do Prof. Dr. Carlos Hairon Ribeiro Gonçalves.}
\definecolor{blue}{RGB}{41,5,195}
\makeatletter
\hypersetup{
   pdftitle={\@title},
	 pdfauthor={\@author},
   pdfsubject={Trabalho de conclusão de curso},
   pdfcreator={pdfLaTeX},
   pdfkeywords={abnt}{latex}{abntex}{abntex2}{trabalho científico},
   colorlinks=true,
   linkcolor=blue,
   citecolor=blue,
   filecolor=magenta,
   urlcolor=blue,
   bookmarksdepth=4
}
\makeatother
\setlength{\parindent}{1.3cm}
\setlength{\parskip}{0.2cm}
\makeindex

\begin{document}
\selectlanguage{brazil}
\frenchspacing
\imprimircapa
\imprimirfolhaderosto*
% \begin{fichacatalografica}
%     \includepdf{fig_ficha_catalografica.pdf}
% \end{fichacatalografica}
% \includepdf{folhadeaprovacao_final.pdf}

\begin{agradecimentos}

\end{agradecimentos}

\setlength{\absparsep}{18pt}
\begin{resumo}
O crescimento da indústria de jogos em diversas áreas - como entretenimento e educação - torna necessário a elaboração de ferramentas que possibilitem a criação destes de forma simples, fácil e objetiva. Por esta razão, o projeto \emph{Ceará Game Tools} (CGT), tem como objetivo possibilitar isso aos usuários que não conhecem os meios e as técnicas envolvidas na criação de jogos.

Então, é proposto nesse trabalho, melhorias importantes para a ferramenta CGT, principalmente, a pré-vizualização do jogo que está sendo construído e também dos seus objetos.

\vspace{\onelineskip}
\noindent
\textbf{Palavras-chave}: CGT, ferramenta;
\end{resumo}

\pdfbookmark[0]{\listfigurename}{lof}
\listoffigures*
\cleardoublepage

\pdfbookmark[0]{\listtablename}{lot}
\listoftables*
\cleardoublepage

\pdfbookmark[0]{\contentsname}{toc}
\tableofcontents*
\cleardoublepage

\textual

\chapter{Introdução} % Ryllari
\label{chap:introducao}
\section{Identificação do problema}
\section{Solução estudada}
\section{Disposição do documento}

\chapter{Problema}
\label{chap:intro}
\section{Módulo anterior da ferramenta}
\section{Escopo das melhorias}
\section{Dificuldades encontradas}
\subsection{Exemplo de jogo}
\section{Game JAM}

\chapter{Descrição das melhorias} % Joel
\label{chap:melhorias}
\section{Pontos de melhoria}
\section{Detalhes da implementação do novo módulo}
\subsection{Métodos e ferramentas}
\subsection{Descrição do módulo}

\chapter{Estudo de caso}
\label{chap:caso}
\section{Análise dos resultados}
\section{Comparativo}
\section{Jogo exemplo}
\section{Projeto de extensão}

\chapter{Conclusão e trabalhos futuros}
\label{chap:conclcsao}

\begin{apendicesenv}
\partapendices
\chapter{Diagramas}
\label{chap:diagramas}

\section{Diagrama de classes}

\begin{tikzpicture}
\begin{umlpackage}{application}
\umlinterface{DrawableObject}{}{
\umlvirt{setB (b : B) : void} \\ getB() : B
}
\end{umlpackage}
\end{tikzpicture}

\section{Diagrama de sequência}

\end{apendicesenv}

\begin{anexosenv}
\partanexos
\chapter{Manual da ferramenta}

\chapter{Código fonte}
\end{anexosenv}
\phantompart
\printindex
\end{document}
