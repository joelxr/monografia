\documentclass[12pt,twoside,openright,a4paper,english,brazil,sumario=tradicional]{abntex2}
\usepackage[utf8]{inputenc}
\usepackage[T1]{fontenc}
\usepackage{indentfirst}
\usepackage{color}
\usepackage{graphicx}
\usepackage{microtype}
\usepackage[brazilian,hyperpageref]{backref}
\usepackage[alf]{abntex2cite}
\usepackage{multicol}
\usepackage{multirow}
\usepackage{enumitem}
\usepackage{amsfonts}
\usepackage{amssymb}
\usepackage{listings}
\usepackage{tikz-uml}

\usepackage{uarial}
\renewcommand{\familydefault}{\sfdefault}
\usepackage{blindtext}

\graphicspath{{../images/}}

\renewcommand{\backrefpagesname}{Citado na(s) página(s):~}
\renewcommand{\backref}{}
\renewcommand*{\backrefalt}[4]{
	\ifcase #1 %
		Nenhuma citação no texto.%
	\or
		Citado na página #2.%
	\else		Citado #1 vezes nas páginas #2.%
	\fi}%

\titulo{Módulo de pré-vizualização para a ferramenta CGT - Ceará Game Tools}
\author{Joel Xavier Rocha\thanks{\texttt{joelxr@gmail.com}}}
\orientador{Prof. Dr. Carlos Hairon Ribeiro Gonçalves}
\local{Fortaleza}
\data{2015}
\instituicao{
  Instituto Federal de Ciência, Educação e Tecnologia do Ceará
  \par
  Engenharia de Computação}
\tipotrabalho{Monografia}
\preambulo{Trabalho de conclusão de curso apresentado à banca examinadora do Instituto Federal de Educação, Ciência e Tecnologia do Ceará para obtenção do grau de bacharel em Engenharia de Computação sob a orientação do Prof. Dr. Carlos Hairon Ribeiro Gonçalves.}
\definecolor{blue}{RGB}{41,5,195}
\makeatletter
\hypersetup{
   pdftitle={\@title},
	 pdfauthor={\@author},
   pdfsubject={Trabalho de conclusão de curso},
   pdfcreator={pdfLaTeX},
   pdfkeywords={abnt}{latex}{abntex}{abntex2}{trabalho científico},
   colorlinks=true,
   linkcolor=blue,
   citecolor=blue,
   filecolor=magenta,
   urlcolor=blue,
   bookmarksdepth=4
}
\makeatother
\setlength{\parindent}{1.3cm}
\setlength{\parskip}{0.2cm}
\makeindex

\begin{document}
\selectlanguage{brazil}
\frenchspacing
\imprimircapa
\imprimirfolhaderosto*

% \begin{fichacatalografica}
%     \includepdf{fig_ficha_catalografica.pdf}
% \end{fichacatalografica}

% \includepdf{folhadeaprovacao_final.pdf}

\begin{agradecimentos}

\end{agradecimentos}

\setlength{\absparsep}{18pt}
\begin{resumo}
O crescimento da industria de jogos em diversas áreas - como entretenimento e educação - torna necessário a elaboração de ferramentas que possibilitem a criação destes de forma simples, fácil e objetiva. Por esta razão, o projeto \emph{Ceará Game Tools} (CGT), tem como objetivo possibilitar isso aos usuários que não conhecem os meios e as técnicas envolvidas na criação de jogos.

Então, é proposto nesse trabalho, melhorias importantes para a ferramenta CGT, principalmente, a pré visualização do jogo que está sendo construído e também dos seus objetos.

\vspace{\onelineskip}
\noindent
\textbf{Palavras-chave}: CGT, ferramenta;
\end{resumo}

\pdfbookmark[0]{\listfigurename}{lof}
\listoffigures*
\cleardoublepage

\pdfbookmark[0]{\listtablename}{lot}
\listoftables*
\cleardoublepage

\pdfbookmark[0]{\contentsname}{toc}
\tableofcontents*
\cleardoublepage

\textual

\chapter{Introdução} % Ryllari
\label{chap:introducao}

% introduzindo sobre o que é o trabalho
Neste trabalho, é apresentado problemas existentes e pontos de melhorias na primeira versão da ferramenta de construção de jogos do projeto CGT (\texttt{ferramenta 1.0}). Bem como, de que modo essas questões foram tratadas e implementadas na segunda versão (\texttt{ferramenta 2.0}). A nova versão da ferramenta tem como objetivo melhorar o processe de criação de jogos e tornar essa tarefa mais intuitiva para o usuário final.

\section{O projeto CGT}
% Descrição/apresentação do projeto CGT
O projeto CGT que surgiu para tornar possível a qualquer pessoa a criação de jogos, assim como é explicado no trabalho \cite{monografia:aquino}, vem sido desenvolvido desde 2013 e possibilita a construção de jogos eletrônicos por qualquer pessoa, não sendo necessário conhecimentos em programação ou nas técnicas de criação de jogos. De foma fácil, simples e prática, a ferramenta, atende a necessidade da maior parte dos usuários, podendo produzir jogos de vários temas e para diversos fins, desde entretenimento a educação, por exemplo. Assim como, é proposto no \emph{website} do projeto que explica, detalhadamente, o seu objetivo.

\begin{citacao}
O projeto Ceará Game Tools tem como objetivo oferecer uma ferramenta para a construção de jogos. Em suma, qualquer um poderá criar seu próprio jogo utilizando componentes \emph{drag-and-drop} e os configurando. \cite{website:projeto-cgt}
\end{citacao}

\subsection{Vínculo com o CNPQ}
A idealização e o início do desenvolvimento da ferramenta CGT deu-se no ano de 2013, através do projeto Ceará Game Tools (CGT) – Pesquisa, Desenvolvimento e Comercialização de Games Temáticos da Cultura Cearense, que é uma inciativa do professor do IFCE Carlos Hairon Ribeiro Gonçalves  apoiado pelo CNPQ\footnote{Centro Nacional de desenvolvimento científico e tecnológico.} por meio do edital 80 de 2013 de número de processo 409227/2013-7.

A proposta deste projeto era desenvolver uma ferramenta de software livre para o desenvolvimento automático de \emph{games} casuais em que qualquer pessoa com conhecimentos medianos de operação de microcomputadores poderia desenvolver jogos. Neste caso, não há necessidade de codificação de software utilizando-se linguagens de programação, mas somente o desenvolvimento de modelos abstratos com o uso de aplicativos de software de uso livre. Além disso, ela pode gerar jogos que podem ser executados em vários sistemas operacionais(\emph{Windows}, \emph{Linux}, \emph{MacOS} e \emph{Android}). Desse modo, os desenvolvedores podem aferir retorno financeiro com a venda e/ou popularização dos jogos, democratizando este canal para todas as pessoas que queiram ingressar nessa área.

\subsection{Trabalhos anteriores}
A partir do projeto vinculado ao CNPQ, a primeira versão da ferramenta foi produzida, trabalhando o desenvolvimento dos jogos apenas com entrada e manipulação de dados, utilizando apenas botões e comandos, sem componentes \emph{drag-and-drop} ou módulo de pré-visualização. Com base no que foi desenvolvido ao longo desta etapa, foi produzido um trabalho acadêmico de conclusão de curso por título “Ceará Game Tools: Uma ferramenta de software livre para geração automática de games” \cite{monografia:aquino}.

Em aspectos gerais, esse trabalho apresenta todo o processo de criação da ferramenta Ceará Game Tools (CGT), que possui um ambiente de desenvolvimento visual simples e intuitivo e uma biblioteca de imagens e sons disponibilizados para os usuários desfrutarem. Além disso, nesse trabalho, relata trabalhos e \emph{softwares} similares ao CGT, descreve os métodos e ferramentas utilizados para o desenvolvimento da aplicação, além da forma com ela foi desenvolvida, os módulos em que o CGT está dividido, os pacotes e classes que compõem a ferramenta, seus diagramas, benefícios, bem como trabalhos futuros e possíveis melhorias para a ferramenta. Nesta última parte, o autor destaca a importância da criação de novas políticas que aumentem as opções dos jogos e implementem novas estruturas, do aumento no número de plataformas de exportação dos jogos desenvolvidos, além de adicionar módulo de pré-visualização do jogo, o qual se concentra o assunto do presente trabalho.

\subsection{Os módulos desenvolvidos para o projeto}
% ferramenta de construção de jogos
O projeto é composto por módulos que dividem entre sí as funcionalidades existentes, esses módulos são mostrados e descritos na tabela \ref{table:modulos}. O módulo da ferramenta é o objeto de estudo para este trabalho e o objetivo deste módulo é implementar toda a interação com o usuário e ser responsável por apresentar o jogo a ele no momento da criação, tudo isso de forma intuitiva, simples e prática. Logo, pode-se notar que, esse módulo é essencial para o projeto e deve ser capaz de lidar com tudo que torna possível a criação de jogos.

\begin{table}[h]
\centering
\begin{tabular}{ | l | p{10cm} | }
    \hline
    Módulos & Descrição \\ \hline
    \texttt{core} & Contém os objetos essenciais para o projeto. Assim como: Ator, Inimigo, Bônus. Na secção \ref{sec:objetos} pode ser visto todos os objetos. \\ \hline
    \texttt{desktop} & Responsável por possibilitar a execução do jogo no computador, em ambientes que rodam os sistemas \emph{Windows}, \emph{Linux} e \emph{MacOS}. \\ \hline
		\texttt{ferramenta} & Corresponde ao módulo objeto de estudo deste trabalho que possibilita a criação dos jogos. \\ \hline
		\texttt{android} & Permite exportar o jogo para execução nos dispositivos que rodam o sistema operacional \emph{Android}. \\ \hline
		\texttt{ios} & Permite exportar o jogo para execução nos dispositivos que rodam o sistema operacional \emph{iOS}. \\ \hline
\end{tabular}
\caption{Módulos existentes no projeto CGT.}
\label{table:modulos}
\end{table}

\section{Introdução ao problema}
\label{sec:intro-problema}
A identificação dos problemas e pontos de melhoria para a ferramenta foram percebidos, principalmente, atráves do uso e estão relacionados a interação com o usuário final e como o jogo em produção é percebido.

A \texttt{ferramenta 1.0} é bastante robusta e atende o propósito de criar os objetos que compõem o jogo e, eventualmente, produzi-lo em forma de aplicação para a plataforma destino. Entretanto, a mesma carece em fornecer algum \emph{feedback} no momento da criação, obrigando ao usuário a sempre executar o jogo para poder perceber atributos essenciais dos objetos dele, tais como: a posição, o tamanho, a textura.

Por exemplo, sejam as posições de um novo objeto configuradas da forma que se deseja, logo após isso, naturalmente, o usuário executa o jogo para poder visualizar o que configurou, e também entender o que representa graficamente o número que inseriou. Pode-se ver esse processo ocorrendo na imagem \ref{fig:intro-problema-1}, as duas janelas que estão abertas são, respectivamente, a do jogo em execução e da ferramenta de criação. Portanto, podemos assumir que para todo objeto recém configurado, será preciso executar o jogo para ver o resultado da configuração. Ou seja, o processo de criação se torna bastante repetitivo.

\begin{figure}[h]
\label{fig:intro-problema-1}
\centering
\includegraphics[width=0.7\textwidth]{images/problema-1.jpg}
\caption{Problema de pré-visualização na \texttt{ferramenta 1.0}.}
\end{figure}

% mais problemas
Além disso, os controles que fazem parte da primeira versão da ferramenta, tornam, na maioria das vezes, oneroso o processo de configuração dos objetos. Por exemplo, na configuração das animações que possui um objeto é necessário criar as animações olhando no \emph{spritesheet}\footnote{Imagem que contém todas as faces de um objeto no jogo.} do objeto e preenchendo campos de texto com o valor da linha e coluna do \emph{sprite} correspondente a animação configurada (ver imagem \ref{fig:intro-problema-2}).

\begin{figure}[h]
\label{fig:intro-problema-2}
\centering
\includegraphics[width=0.7\textwidth]{images/problema-2.jpg}
\caption{Problema com os controles na configuração de uma animção na \texttt{ferramenta 1.0}.}
\end{figure}

Os problemas encontrados na \texttt{ferramenta 1.0} foram a motivação para esse trabalho e serão descritos detalhadamente no capitulo \ref{chap:problemas}.

\section{Disposição do documento}
Este documento segue o método que consiste em: apresentar um problema, propor sua solução e, por fim, estudar as consequências da solução adotada. Dessa forma, tudo estará dividido basicamente nas três partes seguintes:
\begin{description}
	\item[Definição do problema] Mostrado no capítulo \ref{chap:problemas} que tem como objetivo enumerar detalhadamente os problemas e pontos de melhoria encontrados na \texttt{ferramenta 1.0} e como foram encontrados. Consequentemente, prepara-se uma lista das principais funcionalidades que farão parte da nova versão da ferramenta.
	\item[Proposta de melhorias] Corresponde ao capítulo \ref{chap:melhorias} que visa propor soluções para as questões encontradas anteriormente. Além disso, explicá-las e justifica-las detalhadamente, por fim demonstrando a solução adotada e como elas foram alcançadas.
	\item[Análise dos resultados] É um estudo de caso, mostrado no capítulo \ref{chap:caso}, com a nova versão da ferramenta, onde busca-se comparar o tempo de criação de jogo, a facilidade de uso e a quantidade de esforço necessário para configurar.
\end{description}

\chapter{Problema}
\label{chap:problemas}
% breve descrição do problema do preview
A ferramenta de construção de jogos do projeto CGT é robusta e atende aos objetivos propostos. Contudo, ela possui lacunas quando se trata da interação humano-computador (IHC), onde é possível enumerar os seguintes pontos:

\begin{alineas}
	\item A falta de pré-vizualização dos objetos contidos no jogo, que dificulta a percepção do usuário, obrigando-o a sempre executar o jogo para observar as configurações que foram feitas;
	\item Os controles não são claros e, assim, produzir um jogo pode acabar sendo uma experiência onerosa, confusa e repleta de configurações repetitivas.
\end{alineas}

Com esses itens em mente, mostra-se como construir um jogo na primeira versão da ferramenta e, a partir daí, quais problemas e melhorias são notados.

\section{Objetos de um jogo}
\label{sec:objetos}

A fim de enumerar os problemas e melhorias, torna-se necessário construir um jogo com a \texttt{ferramenta 1.0} e, para fazê-lo, deve-se conhecer os objetos que existem e como devemos configura-los. Então, a seguir, há uma lista dos possíveis objetos que podem compor um jogo:

\begin{alineas}
	\item Mundo,
	\item Ator,
	\item Inimigo,
	\item Opositor,
	\item Bônus,
	\item Projétil,
	\item Tela,
	\item Botão de tela,
	\item Vida de um objeto e
	\item Munição de um objeto.
\end{alineas}

Na secção \ref{sec:dificuldades}, será criado um jogo de exemplo na ferramenta, a fim de mostrar os problemas e pontos de melhoria percebidos. O manual da \texttt{ferramenta 2.0} está na parte de anexos e pode ser consultado para entender como funciona a criação de cada objeto listado.

\section{Dificuldades encontradas}
\label{sec:dificuldades}
% detalhar dificuldades com figuras da ferramenta 1.0, fazer um jogo de exemplo que será usado posteriormente
Nesta secção vamos criar um jogo com cada componente existente na ferramenta e enumerar os problemas encontrados. Notar que, a \texttt{ferramenta 1.0} está disponível no \href{http://www.cgt.ifce.edu.br/downloads.php}{site do projeto CGT} na secção de \emph{downloads}\footnote{http://www.cgt.ifce.edu.br/downloads.php}.

\subsection{A tela inicial}

\begin{figure}[htb]
\label{fig:dificuldades-1}
\centering
\includegraphics[width=0.7\textwidth]{images/dificuldades-1.jpg}
\caption{Tela inicial da \texttt{ferramenta 1.0}.}
\end{figure}

\subsection{Abas como agregadores}

\subsection{Organização dos objetos}

\subsection{Paineis de configuração}

\section{Game JAM} % Ryllari

\chapter{Descrição das melhorias} % Joel
\label{chap:melhorias}
\section{Pontos de melhoria}
\section{Detalhes da implementação do novo módulo}
\subsection{Métodos e ferramentas}
\subsection{Descrição do módulo}

\chapter{Estudo de caso}
\label{chap:caso}
\section{Análise dos resultados}
\section{Comparativo}
\section{Jogo exemplo}
\section{Projeto de extensão}

\chapter{Conclusão e trabalhos futuros}
\label{chap:conclcsao}

\postextual
\bibliography{references}

\begin{apendicesenv}
\partapendices
\chapter{Diagramas UML}
\label{chap:diagramas}

\section{Diagrama de classes}
No diagrama de classes a seguir, temos a notação UML para as classes do módulo \texttt{ferramenta} do projeto CGT que estão localizadas dentro do pacote \texttt{br.edu.ifce.cgt.application}.

\begin{figure}[h]
\label{fig:diagrama-classes}
\centering
\begin{tikzpicture}
\begin{umlpackage}{vo}
\umlinterface{DrawableObject}{}
{
	getObject() : T \\
	setObject(obj : T) : void \\
	getPane() : Node \\
	drawObject() : void \\
	drawConfigurationPanel() : void \\
	onCreate() : void \\
	onStart() : void \\
	destroy() : boolean }
\umlclass[x=8,y=0]{AbstractDrawableObject}
{
	- object : T \\
	- drawableObjectPane : Pane \\
	- drawConfigurationPanel : Pane
}
{
	getDrawableObjectPane() : Pane \\
	getDrawableConfigurationsPane() : Pane \\
	updateDrawPane(node : Node) : void \\
	updateConfigPane(node : Node) : void \\
	updateConfigPane(pane : Pane) : void
}
\umlclass[x=0,y=-6]{CGTProjectDrawable}
{
	- size : Rectangle \\
	- projectPane : ConfigProjectPane
}{}
\umlclass[x=8,y=-6]{CGTGameObjectDrawable}
{
	- gameObjectTitledPane : GameObjectPane \\
	- worldName : String \\
	- bounds : Rectangle \\
	- collision : Rectangle \\
	- preview : Draggable
}
{}
\umlclass[x=0,y=-9]{CGTGameActorDrawable}{}{}
\umlclass[x=8,y=-10]{CGTGameEnemyDrawable}{}{}
\umlclass[x=0,y=-11]{CGTGameOppositeDrawable}{}{}
\umlclass[x=8,y=-12]{CGTGameBonusDrawable}{}{}
\umlclass[x=0,y=-13]{CGTGameProjectitleDrawable}{}{}
\umlclass[x=8,y=-14]{CGTGameScreenDrawable}{}{}
\umlclass[x=0,y=-15]{CGTButtonScreenPreview}{}{}
\umlclass[x=8,y=-16]{CGTLifeBarDrawable}{}{}
\umlclass[x=0,y=-17]{HUDComponetDrawable}{}{}
\umlclass[x=8,y=-18]{DrawableObjectTreeCellImpl}{}{}
%\umlimpl{DrawableObject}{AbstractDrawableObject}
%\umlimpl{AbstractDrawableObject}{CGTProjectDrawable}
%\umlimpl{AbstractDrawableObject}{CGTGameObjectDrawable}
%\umlimpl{CGTGameObjectDrawable}{CGTGameActorDrawable}
%\umlimpl{CGTGameObjectDrawable}{CGTGameEnemyDrawable}
%\umlimpl{CGTGameObjectDrawable}{CGTGameOppositeDrawable}
%\umlimpl{CGTGameObjectDrawable}{CGTGameBonusDrawable}
%\umlimpl{CGTGameObjectDrawable}{CGTGameProjectitleDrawable}
\end{umlpackage}
\end{tikzpicture}
\caption{Diagrama de classes UML para o módulo implementado.}
\end{figure}

\end{apendicesenv}

\begin{anexosenv}
\partanexos
\chapter{Manual da ferramenta}

\chapter{Código fonte}
\end{anexosenv}
\phantompart
\printindex
\end{document}
